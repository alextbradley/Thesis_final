\begin{abstract}
In this thesis, we study bendotaxis -- a novel, passive droplet transport mechanism driven by the coupling between capillary pressure within the droplet and bending deformations of the channel confining it. This mechanism is inspired by observations of the legs of Gerris Regimis, which are covered in setae rendering them superhydrophobic; the setae deform by bending in response to the capillary pressure of vapour droplets as they condense between setae, resulting in their dynamic expulsion, and thus helping to maintain a superhydrophobic leg.

In the first half of this thesis, we focus on two dimensional scenarios to isolate the underlying competition between bending and capillarity. In this setup, the direction of droplet transport is, surprisingly, always the same, regardless of the wettability of the channel. We use a combination of experiments at a macroscopic scale and a simple mathematical model to study this motion, focusing in particular on the time scale associated with the motion. We make observations that may inform the design of superhydrophobic surfaces exploiting bendotaxis for anti-fogging and also investigate two physical effects -- contact angle hysteresis and droplet self-trapping -- which can impede droplet transport by bendotaxis. 

In the second half of this thesis, we focus on the influence of channel geometry on droplet transport by bendotaxis.  Further motivation is provided by experiments in which a regular, periodic pattern emerges as liquid droplets are condensed into deformable microchannels and move in a manner similar to bendotaxis. We hypothesize that this wavelength selection is the result of a novel instability, reminiscent of the Rayleigh-Plateau instability but mediated by channel elasticity. By developing and analyzing a mathematical model of a system susceptible to this instability, we study its characteristic behaviour and the consider the effect of adding liquid to the channel as the instability develops.

Finally, we consider how collaborative effects of droplet transport by bendotaxis in neighbouring channels may lead to clustering. We first develop a mathematical model for an array of droplet-channel systems, each of which would undergo bendotaxis in isolation, and, using this, identify two distinct regimes for the cluster sizes in the limits of weak and strong surface tension, which are then derived analytically using asymptotic methods. 
\end{abstract}