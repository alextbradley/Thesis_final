\chapter{Epilogue}
In this thesis, we have presented a detailed study of bendotaxis -- a new mechanism by which a droplet squeezes itself out of an elastic channel: the capillary pressure of the droplet bends the channel walls confining the droplet in such a way that causes the droplet to move spontaneously.  We have focussed on questions pertinent to the possible application of bendotaxis to self cleaning surfaces, which has motivated consideration of the time scale of motion, physical effects that might impede droplet transport, the influence of channel geometry, and the collaborative effects that arise between neighbouring channels. In this chapter we summarize our results and suggest some ideas for further work.

\section{Summary of the thesis}
We began in Chapter 1 with a discussion of the mechanism of bendotaxis: the Laplace pressure within a droplet induces a tapering of the confining channel that drives it towards the free end. By considering this mechanism, we predicted that the direction of motion is universal -- droplets should be transported to the free end, regardless of whether they wet the channel or not -- and verified this with a proof-of-concept experiment.

Having shown that the direction of bendotaxis is universal, we turned in Chapter~2 to develop a two-dimensional dynamic model of bendotaxis. The physical processes represented by our model were motivated by the proof of concept experiment of Chapter~1; in particular, we exploited the small aspect ratio of the droplet and the slenderness of the channel walls to reduce the dimensionality of the problem. The resulting spatially one-dimensional system of PDEs that describe this model were solved numerically; using these solutions, we verified that wettability independent droplet transport is a feature that is reproduced by our model and identified three key characteristic features of the dynamics of bendotaxis in two-dimensional channels: Firstly, motion occurs on two time scales: a short time scale on which the channel walls respond to a torque imbalance by bending rapidly (causing the droplet to spread), and a longer time scale on which droplets translate to the free end. Secondly, even in the absence of inertia, droplets accelerate along the channel, in contrast to bendo-capillary imbibition from a fixed bath of liquid, for which the leading meniscus decelerates throughout the motion. Finally,  when surface tension is sufficiently strong, the channel walls touch before the droplet reaches the free end. We made analytic progress in understanding the first two of these observations by assuming small channel wall deflections;  this asymptotic analysis suggested that the time scale of bendotaxis is inversely proportional to the product of the channel bendability $\nu$ and the droplet volume $V$, as a scaling argument valid for small droplets also suggested. The mathematical model provides some insight that could inform the design of superhydrophobic surfaces  that exploit bendotaxis for anti-fogging. In particular, we noted that more compliant channels will transport droplets faster (it would be desirable to remove droplets from such a surface as quickly as possible), but this faster transport must be balanced with the increased risk of trapping the droplets within the channel, thereby compromising the surface’s performance. 

In Chapter 3, we presented an experimental study of bendotaxis. The experimental setup was similar to that in the proof of concept experiments of Chapter 1, except that the parameters of the system (specifically the channel geometry, droplet viscosity, and channel wall stiffness) were systematically varied to present a robust test of the mathematical model developed in Chapter 2. The experimental droplet trajectories have a similar shape and time scale of motion to those predicted by the analysis of our model (both from scaling, asymptotics, and numerics). Nevertheless, the experimentally observed motion was systematically slower than predicted -- a discrepancy that is consistent with the two-dimensional nature of our theory and the  associated overestimation of the relative droplet volume. To facilitate comparison of a much larger data set and reduce dependence on initial conditions, we introduced the time taken to traverse a section of the channel as a proxy for the full dynamics represented by the trajectories. Analysis of the experimental data showed that the small volume result of Chapter 2 describes the dynamics reasonably well, even when the droplet has finite volume. Moreover, this analysis showed that deviations from the small volume result -- the non-linearity associated with larger channel deflections, and the finite drop size effects predicted by the model -- are represented in the experimental data in a way that is consistent with the model predictions. 

In Chapter 4, we considered two mechanisms by which droplets might become trapped in channels whilst undergoing bendotaxis. The first of these was geometric trapping, which describes the scenario first encountered in Chapter~2 in which the channel walls touch before the droplet reaches the free end. We built upon the ideas of Chapter~2, where we saw that the channel bendability for which droplet trapping occurs is reasonably insensitive to the droplet volume and initial position, provided that the droplet starts reasonably far from the free end of the channel. In particular, we extended our model of bendotaxis to describe the behaviour beyond the point of wall contact, and identified two post contact  scenarios -- channel walls touching at a single point, and channel walls making contact along a portion of their length, the latter occuring only when the channel bendability $\nu$ is very large. Numerical solutions of the model equations display complex dynamics that result from the strong squeezing at early times that is associated with large channel bendability, as well as from the competition between two diverging quantities (the meniscus pressure and viscous dissipation) as the droplet advances into a channel of vanishing thickness. 
%If, on the one hand, the meniscus approaches the end of the channel while the walls are simply touching, it appears to reach the end in finite time with asymptotic behaviour faster than the corresponding rigid case. On the other hand, if the meniscus approaches the end of a channel whose walls are in contact along a portion of their length (making a cusp at the contact point), the meniscus appears to take an infinite amount of time to reach the cusp.

In the second half of Chapter 4, we considered when droplets may be trapped as a result of contact angle hysteresis. We extended the model to encode the idea of a non-unique contact angle at a stationary interface and supplemented this with the simplest possible contact angle dynamics.  In numerical solutions, we saw that droplets may be trapped (reach equilibria) part way along the channel, but this depends critically on whether the contact angle asymmetry across the droplet that is allowed by the given contact angle hysteresis is sufficient to permit an equilibrium. We mapped out the conditions under which such an equilibrium is possible and found, in particular, that for a given droplet, it may `escape' the channel provided that it starts close enough to the free end. Further, we found that droplets are more likely to be trapped in channels with higher contact angle hysteresis, confirming the need to minimize hysteresis in any superhydrophobic surface that seeks to exploit bendotaxis for anti-fogging. We also verified that the trapped states are indeed equilibria, that they are linearly stable, and that droplets will never translate to equilibrium -- heuristically, once droplets `get going' they will not be stopped.

In Chapter 5, we sought to understand the weaving instability that is observed in experiments of condensation of droplets into deformable microchannels~\citep{Seemann2011JPhysCondMat}. In doing so, we identified a novel bendo-capillary instability that is reminiscient of the Rayleigh-Plateau instability (in that it relies on a competition between the principal interfacial curvatures) and the instability described by~\cite{AlHousseiny2012NaturePhysics} (in that it is mediated by the channel that confines the liquid). Unlike the rigid (Al-Housseiny) case, however, the channel tapering in the bendo-capillary case is set by the liquid pressure meaning that both wetting and non-wetting liquids may, in theory, experience instability in the same channel. To study this bendo-capillary instability, we developed a mathematical model of a three-dimensional system and again exploited linear beam theory and lubrication theory to reduce its dimensionality. We focussed first on the no condensation case, where equilibria are possible. A linear stability analysis revealed that these equilibria are unstable to perturbations of a sufficiently small wavenumber and that the corresponding growth rates are highly sensitive to the amount of liquid in the channel, which is parametrized by the cross-sectional volume $V$. We then considered the effect of a non-zero condensation rate, which changed the picture in two important ways: (i) the cross-sectional volume $V$ becomes time-dependent, and (ii) condensation may drive dynamic effects in the (now time-dependent) base state. We saw that when the condensation rate is small, condensation enters only through the instantaneous value of $V$ and the instantaneous growth rate of a perturbation is given by the corresponding zero-condensation result; as a consequence of the sensitivity of the zero-condensation results to $V$, we saw how modes that grow slowly at first are able to catch up with, and overtake, those that grow faster initially. We speculated that these later modes reach the non-linear regime sooner. Numerical solutions of the linearized equations revealed that this mechanism is also present for condensation rates of any size, but interacts with the condensation driven dynamics. This tends to enhance the instantaneous growth rate and lengthen the band of unstable modes (compared to the quasistatic case) with longer wavelength modes being enhanced preferentially. Despite the simplicity of our model, its predictions agree in their order of magnitude with the condensation experiments of~\cite{Seemann2011JPhysCondMat}.

In Chapter 6, we sought to understand how multiple droplets in neighbouring channels might affect each other's bendotaxis -- a question relevant not only for self cleaning surfaces (which naturally have many channels) but also for the condensation experiments in which the weaving instability occurs simultaneously in neighbouring channels. We developed a mathematical model of this `multi-body’ bendotaxis in which, for simplicity, we assumed that the channel walls are rigid, and are tethered at their base by torsional elastic springs. As part of the model development, we identified the importance of a parameter $\Gamma$  that encodes the ability of the individual droplets to deform the channel that confines them, and is analogous to the bendability $\nu$ used throughout Chapters 2--5. Importantly, this torsional spring model is also able to describe wettability-independent droplet transport in a single channel. Modelling with torsional springs (rather than the potentially more realistic beam theory applied elsewhere in this thesis) simplifies the model equations to a system of ODEs which have an equilibrium corresponding to equal volume droplets in un-deformed channels. A linear stability analysis revealed that this equilibrium is always unstable and we do not, therefore, expect to see it in practice. Moreover, the linear stability analysis suggested that a pairwise mode is always the fastest growing. However, numerical solutions of the governing ODEs with an initial condition close to the equilibrium displayed a range of cluster sizes. For small $\Gamma$, the system almost always selects the pairwise mode (in accord with the linear stability analysis) because the angle of the channel walls (and thus direction of droplet motion) is set simply by the difference in position of droplets across them. We rationalized this result with an analysis that considers the governing ODEs in the case of small deflections that are associated with small $\Gamma$. For large $\Gamma$, however, a distribution of cluster sizes were observed, with a maximal size that scales with $\Gamma^{1/2}$. By using a discrete-to-continuum approximation of the ODEs, we suggested that this appears to be the result of the propagation of fronts through the system which `lock in' clusters of this set size; with a random initial condition, these fronts interact with one another to `smooth out' the cluster sizes between the pairwise mode preferred by the linear stability analysis and the maximal cluster size set by the front propagation. We also considered the implications of this model for self-cleaning surfaces exploiting bendotaxis; the key result is that for larger $\Gamma$, a greater proportion of droplets reach the free end, but they do so in channels that are almost closed at the free end, and thus potentially difficult for droplets to be removed from.



\section{Future work}
In this section, we provide a brief overview of how the work presented in this thesis might be extended.

We hypothesized in Chapter 3 that some of the discrepancies between experimental data and the predictions of the model of Chapter 2 are a result of  the two-dimensional nature of the model, which does not account for the fact that the length of the droplet (and interfacial curvature) vary in the third dimension in the experiments. It would be useful to quantify these effects and thereby assess the validity of this hypothesis. We anticipate that to do so, the mathematical model of Chapter 2 could be extended to describe variations of a finite extent in the third spatial direction (similar to the model of Chapter 5, although there the liquid had infinite extent in the third direction). With geometric extensions to the model of Chapter 2 in mind, one might also consider how the shape of the channel walls influences the dynamics of bendotaxis; in particular, one might take further inspiration from the setae of the water strider, which played an important role in the problem that provided our original motivation for studying bendotaxis. The setae are conically shaped, and they should, therefore, have a lower resistance to deformation at their tip than at their base; this should, in turn, exacerbate the relative `softness' of the free end of the channel, potentially increasing the speed of motion of droplet transport, but increasing the risk of trapping. A first step towards modelling this softening might be to consider the two-dimensional configuration of Chapter 2 with a variable bending stiffness $B = B(x)$. (Indeed, with a variable bending stiffness, many more possibilities open up, such as bendotaxis in tubes, which could be useful for micro-fluidic applications.)

In Chapter 4, we observed numerically that a droplet approaching the end of a channel whose walls are in contact at a single point appear to be able to reach this contact point in finite time (approaching with power law behaviour), but a droplet approaching the end of a channel whose walls are in contact over a portion of their length appear to take an infinite time to do so. It would be good to confirm and rationalize this behaviour, perhaps using an asymptotic approximation of the solution as the droplet approaches the contact point.

We focussed in Chapter 4 on identifying when droplets are trapped with a view to preventing this from happening, but, in other situations, this behaviour could perhaps be turned to our advantage. Indeed,  the trapping of droplets in tapered channels that results from contact angle hysteresis is believed to be part of a wider `capillary ratchet' mechanism that is exploited by feeding shorebirds~\citep{Prakash2008Science}. Furthermore, with modifications to the channel set-up, additional possibilities for droplet control might open up; for example, by controlling tapering angle of the channel walls at the clamped end, we can imagine a scenario in which only droplets with a sufficiently large surface tension are transported to the free end, whilst others are trapped, and others still (of the same wettability) are transported to the clamped end of the channel.

The natural extension to our study of the bendo-capillary instability in Chapter 5 is to solve the corresponding non-linear equations numerically. This would allow us to assess how the linearly unstable modes interact with one another as well as to predict which mode will ultimately be observed at late times, once the linearized analysis is no longer valid. We stress, however, that this is not a simple task for myriad reasons. For example, care is required in discretizing the biharmonic operator in the beam equation on a moving domain, fluid and solid deformations must be solved for simultaneously (leading to a large number of degrees of freedom), and the free surface must be appropriately parametrized.

It would also be interesting to study how the mobility of the liquid affects the bendo-capillary instability; we assumed for simplicity that the liquid is sat at the base of the channel but a similar instability should occur if there is a second meniscus closer to the clamped end (a scenario that would be more appropriate for bendotaxis). In that case, there are potentially two modes of instability, depending on whether the protrusions of the perturbation to the leading meniscus are in phase or out of phase with those at the trailing meniscus (sinusoidal or varicose modes). 

There are several ways in which the study of multi-body bendotaxis in Chapter 6 could be extended. Most pressingly, we assumed that the droplets have a constant volume, and focussed primarily on the case when each droplet has the same volume. This is often not the case in experiments of bendo-capillary clustering (for example, in the condensation experiments which provide motivation). In addition, a recent study by~\cite{Hadjittofis2016JFM} identified how mass changes in a similar system can exert a strong influence on the dynamic behaviour, both suppressing or enhancing clustering depending on how quickly it occurs. One might extend this work further to imagine a scenario in which droplets nearer to the free end are evaporated preferentially; this may exert a stabilizing control on our system, which is always unstable when droplet volumes are constant. 

With respect to self cleaning surfaces that could exploit bendotaxis, it would be useful to consider boundary conditions other than periodic in the $y$ direction because, after all, a real surface would have finite extent. Beyond the channel geometry considered here, one might extend our model to include two dimensional arrays of pillars; key difficulties with this would be that surface tension forces require a more careful calculation (because of the three-dimensional menisci), and resistive forces from the lubricating flow might be reduced because the liquid can flow in two directions.

Finally, we note that in the second part of this thesis, we took much inspiration from the condensation experiments of~\cite{Seemann2011JPhysCondMat}, but made progress towards understanding the observations by decoupling the behaviour into the weaving instability in a single channel and the interaction between bendotaxis in neighbouring two-dimensional channels. For a model that is more faithful to the experiments, one would have to couple these two together (i.e. Chapters 5 and 6), as well as account for droplets and contact angle hysteresis (Chapter 4), all of which would pose a significant modelling challenge.